\usepackage{ctex}
\usepackage{fancyhdr} %用来设置页眉页脚内容、样式等
\usepackage{geometry} %利用该宏包设置页面参数,包括页眉页脚的宽度等
\usepackage{setspace} %设置间距的宏包
\usepackage{ulem} %专门用来设置下划线相关的包,如uline,underline不在其中
\usepackage{graphicx} %用来插入图片的包
\usepackage{titlesec} %用来设置章节标题等的样式,字体,间距等等的包
%\usepackage{abstract} %只能用来设置摘要的特别简单的东西,查看文档,几乎不能灵活设置,所以不常使用
%========对应titlesec包======%
\titleformat*{\section}{\raggedright\setstretch{1.25}\heiti\zihao{-3}\bfseries} %setstretch应该是1.5,但是不知道为什么1.5宽了
\titlespacing*{\section}{0pt}{0pt}{17.385pt}
%==========end===========%
%============重新定义Abstract的格式===========%
\renewenvironment{abstract}{%
	\par\zihao{-4}%指定段落字体大小
	\vspace*{-12.5pt}\noindent\mbox{}\hfill{\bfseries\zihao{-3}\heiti\abstractname}\hfill\mbox{}\par %\vspace{-12.5pt}不知道什么原因,headsep就是页眉到正文的距离应该是固定30pt,但是摘要却低于30pt,所以加入vspace,但是vspace在首行不起作用,所以用vspace*,但是vspace*直接空两行,所以用负距离缩减,具体只能测量长度。
	\vskip 17.385pt} %是标题距正文的距离
%================Abstract end============%

\setCJKmainfont{宋体} %与\fontspec命令对应,只不过fontspec是指定英文的
\setmainfont{Times New Roman} %设置默认英文字体看,使得中文字体不更改英文ziti
\geometry{top=3.92cm, bottom=3.55cm, left=2.5cm, right=2.5cm, headheight=10.5pt, headsep=30pt, footskip=1.55cm, heightrounded}
%对应word的top=3.5,bottom=2.5,left=2.5,right=2.5,页眉顶端距离=2.5cm,页脚底端距离=2cm

%====================字体命令设置部分=============================%
%=====楷体GB2312设置=====%
\setCJKfamilyfont{kaiGB}[AutoFakeBold={3}]{楷体_GB2312} 
\newcommand{\kaiGB}{\CJKfamily{kaiGB}} %将命令重新定义,方便以后使用
%将楷体GB2312用kaiGB重命名,AutoFakeBold=3指调用\textbf加粗时字体多粗,默认值为3
%当不调用\textbf{}时,是不会加粗的
\setCJKfamilyfont{songti}[AutoFakeBold={3}]{SimSun}
\renewcommand{\songti}{\CJKfamily{songti}}
\setstretch{1.33} %设置全篇的行间距默认为1.33近似word中的1.25
%====================end=========================================%
%============以下部分都是fancyhdr包中的内容=======================%
\pagestyle{fancy}
\fancyhead[CO]{\zihao{5}{论文中文题目}}
\fancyhead[CE]{\zihao{5}{兰州交通大学硕士学位论文}}
% 以下四句保证中有中间页眉和页码,不然有时候会出现别的章节的页眉
\lhead{}
\rhead{}
\lfoot{}
\rfoot{}
%\cfoot{-\thepage-} %设计页码的风格样式