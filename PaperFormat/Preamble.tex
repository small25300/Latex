\usepackage{ctex}
\usepackage{fancyhdr} %用来设置页眉页脚内容、样式等
\usepackage{geometry} %利用该宏包设置页面参数,包括页眉页脚的宽度等
\usepackage{setspace} %设置间距的宏包
\usepackage{ulem} %专门用来设置下划线相关的包,如uline,underline不在其中
\usepackage{graphicx} %用来插入图片的包
\usepackage{titlesec} %用来设置章节标题等的样式,字体,间距等等的包
\usepackage{titletoc} %用来设置目录格式,并不是生成目录的
\usepackage[backref=true]{hyperref}
\usepackage{bibspacing} % 需要单独下载放在当前目录下。

\hypersetup{hidelinks} %隐藏超链接中的红色的框

%@@@@@@@@@@@@@@@@@@@@@@@@@@@@@@@@@@@@@@@@@@@@@@@@@@@@@@@@@@@@@@@@@@@@@@@@@%
%==================================对应titlesec包=======================%
\titlelabel{\thetitle \hspace{0.5em}} %设置所有section、subsection、subsubsection序号与标题之间的空格间距
\titleformat*{\section}{\raggedright\setstretch{1.25}\zihao{-3}\heiti} %setstretch应该是1.5,但是不知道为什么1.5宽了
\titlespacing*{\section}{0pt}{0pt}{17.385pt}
%=================subsection===================%
\titleformat*{\subsection}{\vspace{0.5pt}\raggedright\setstretch{1.5}\heiti\zihao{4}\bfseries}
%===================subsubsection=============%
\titleformat*{\subsubsection}{\vspace{0.5pt}\raggedright\setstretch{1.5}\heiti\bfseries}
%========================================end==========================%
%@@@@@@@@@@@@@@@@@@@@@@@@@@@@@@@@@@@@@@@@@@@@@@@@@@@@@@@@@@@@@@@@@@@@@@@@@%
%============用titletoc包中的命令设置section的目录格式=============%
\titlecontents{section} %titletoc中用来设置目录格式的命令
[1em] %是目录内容(不包括标签即序号)距离文档body左边距的距离(页面文字不缩进的开头位置)
%因为目录内容距离body左边有一个序号(1、2、3等)和一个空格所以是1em,序号占0.5em,空格占0.5em,合起来就是1em
{}
{\contentslabel{1em}} % 在目录label还没有放置之前,空出的空间,然后再此空间的开头放置label
%注意:不是label距离目录内容的距离,而是label距离内容的距离+label所占的宽度
{\hspace{-1em}} %没有标签序号的目录的使用,使得不带序号的目录与带序号的目录的序号对其。
{\titlerule*[3.5pt]{.}\hspace{-1em}\contentspage} %此处的\hspace{}用来控制连线点与页码的距离

%=================section设置end========================%

%============用titletoc包中的命令设置subsection的目录格式=============%
\titlecontents{subsection}[3.8em]{}{\contentslabel{1.8em}}{}{\titlerule*[3.5pt]{.}\hspace{-1em}\contentspage} 
%=================subsection设置end========================%

%============用titletoc包中的命令设置subsubsection的目录格式=============%
\titlecontents{subsubsection}[6.4em]{}{\contentslabel{2.6em}}{}{\titlerule*[3.5pt]{.}\hspace{-1em}\contentspage} 
%=================subsubsection设置end========================%
%======================目录标题格式的设置=======================%
\renewcommand{\contentsname}{\vspace*{-1.2em}\centerline{目\hspace{2em}录}\vspace{-2.7em}} %让目录标题居中,vspace{}指目录标题距离目录内容的垂直距离。
%=====================目录标题格式设置end================%
%@@@@@@@@@@@@@@@@@@@@@@@@@@@@@@@@@@@@@@@@@@@@@@@@@@@@@@@@@@@@@@@@@@@@@@@@@@@@%
%================自定义environment===================================%

%============自定义AbstractCH=============%
\newenvironment{AbstractCH}
{\begin{center}\setstretch{1.5}\bfseries\heiti\zihao{-3} 摘\hspace{2em}要 \vskip 17.385pt\end{center}\par\zihao{-4}} %\end{center}后的\par\zihao{-4}指标题后面内容另起一段,字体为小四
{\vspace{3ex}\par\bfseries\noindent\heiti\textbf{关键字:}
\vspace{3ex}\par\bfseries\noindent\heiti\textbf{论文类型:}} %以关键字为结尾
%==============AbstractCh end=============%

%============自定义AbstractEN=============%
\newenvironment{AbstractEN}
{\begin{center}\setstretch{1.5}\bfseries\heiti\zihao{-3} Abstract \vskip 17.385pt\end{center}\par\zihao{-4}} %\end{center}后的\par\zihao{-4}指标题后面内容另起一段,字体为小四
{\vspace{3ex}\par\bfseries\noindent Key Words:} %以关键字为结尾
%==============AbstractCh end=============%

%============自定义Conclusion=============%
\newenvironment{Conclusion}
{\begin{center}\setstretch{1.5}\bfseries\heiti\zihao{-3} 结\hspace{2em}论 \vskip 17.385pt\end{center}\par\zihao{-4}} %\end{center}后的\par\zihao{-4}指标题后面内容另起一段,字体为小四,没有指定Conclusion环境的结束,因为不需要
%==============Conclusion end=============%

%==============自定义Thanks==============%
\newenvironment{Thanks}
{\begin{center}\setstretch{1.5}\bfseries\heiti\zihao{-3} 致\hspace{2em}谢 \vskip 17.385pt\end{center}\par\zihao{-4}} %\end{center}后的\par\zihao{-4}指标题后面内容另起一段,字体为小四
%==============Thanks end=============%

%==============自定义Reference=========%
%\newenvironment{Reference}
%{\begin{center}\setstretch{1.5}\bfseries\heiti\zihao{-3} 参\hspace{1em}考\hspace{1em}文\hspace{1em}献 \vskip 17.385pt\end{center}\par\zihao{-4}} %\end{center}后的\par\zihao{-4}指标题后面内容另起一段,字体为小四
%%%参考文献有自己的定义标题及相应的格式,不需要重新定义
%==============自定义Appendix==============%
\newenvironment{Appendix}
{\begin{center}\setstretch{1.5}\bfseries\heiti\zihao{-3} 附录A~~我的附录内容 \vskip 17.385pt\end{center}\par\zihao{-4}} %\end{center}后的\par\zihao{-4}指标题后面内容另起一段,字体为小四
%==============Appendix end=============%

%================================自定义end================================%
%@@@@@@@@@@@@@@@@@@@@@@@@@@@@@@@@@@@@@@@@@@@@@@@@@@@@@@@@@@@@@@@@@@@@@@@@@@@@%
%\setCJKmainfont{宋体}前面必须空一行,才可以,不然出错,具体原因不详。

\setCJKmainfont{宋体} %与\fontspec命令对应,只不过fontspec是指定英文的
\setmainfont{Times New Roman} %设置默认英文字体看,使得中文字体不更改英文ziti
\geometry{top=3.92cm, bottom=3.55cm, left=2.5cm, right=2.5cm, headheight=10.5pt, headsep=30pt, footskip=1.55cm, heightrounded}
%对应word的top=3.5,bottom=2.5,left=2.5,right=2.5,页眉顶端距离=2.5cm,页脚底端距离=2cm

%====================字体命令设置部分=============================%
%=====楷体GB2312设置=====%
\setCJKfamilyfont{kaiGB}[AutoFakeBold={3}]{楷体_GB2312} 
\newcommand{\kaiGB}{\CJKfamily{kaiGB}} %将命令重新定义,方便以后使用
%将楷体GB2312用kaiGB重命名,AutoFakeBold=3指调用\textbf加粗时字体多粗,默认值为3
%当不调用\textbf{}时,是不会加粗的
\setCJKfamilyfont{songti}[AutoFakeBold={3}]{SimSun}
\renewcommand{\songti}{\CJKfamily{songti}}
\setstretch{1.33} %设置全篇的行间距默认为1.33近似word中的1.25
%====================end=========================================%
%@@@@@@@@@@@@@@@@@@@@@@@@@@@@@@@@@@@@@@@@@@@@@@@@@@@@@@@@@@@@@@@@@@@@@@@@@@@@%
%============以下部分都是fancyhdr包中的内容=======================%
\pagestyle{fancy}
\fancyhead[CO]{\zihao{5}{论文中文题目}}
\fancyhead[CE]{\zihao{5}{兰州交通大学硕士学位论文}}
% 以下四句保证中有中间页眉和页码,不然有时候会出现别的章节的页眉
\lhead{}
\rhead{}
\lfoot{}
\rfoot{}
%\cfoot{-\thepage-} %设计页码的风格样式
%@@@@@@@@@@@@@@@@@@@@@@@@@@@@@@@@@@@@@@@@@@@@@@@@@@@@@@@@@@@@@@@@@@@@@@@@@@@@%
%================参考文献引用==========%
\renewcommand{\refname}{\centerline {参\hspace{1em}考\hspace{1em}文\hspace{1em}献}\vskip -22pt} %距离参考文献列表的宽度,使得与结论等一样
\bibliographystyle{MyBibStyle}
%\addbibresource{Contents/Bibliography.bib}