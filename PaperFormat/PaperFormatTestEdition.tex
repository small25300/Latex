\documentclass[a4paper]{article}
% 引入的宏包
\usepackage{ctex}
\usepackage{fancyhdr} %用来设置页眉页脚内容、样式等
\usepackage{geometry} %利用该宏包设置页面参数,包括页眉页脚的宽度等
\usepackage{setspace} %设置间距的宏包
\usepackage{ulem} %专门用来设置下划线相关的包

\setCJKmainfont{宋体} %与\fontspec命令对应,只不过fontspec是指定英文的
\setmainfont{Times New Roman} %设置默认英文字体看,使得中文字体不更改英文ziti
\geometry{top=3.5cm, bottom=2.5cm, left=2.5cm, right=2.5cm, includeheadfoot}

%====================字体命令设置部分=============================%
%=====楷体GB2312设置=====%
\setCJKfamilyfont{kaiGB}[AutoFakeBold={3}]{楷体_GB2312} 
%将楷体GB2312用kaiGB重命名,AutoFakeBold=3指调用\textbf加粗时字体多粗,默认值为3
%当不调用\textbf{}时,是不会加粗的
\newcommand{\kaiGB}{\CJKfamily{kaiGB}} %将命令重新定义,方便以后使用
%====================end=========================================%

%============以下部分都是fancyhdr包中的内容=======================%
%\pagestyle{fancy}
%\lhead{}
%\chead{兰州交通大学}
%\rhead{}

\begin{document}
	\newgeometry{top=2cm, bottom=2cm, left=3.7cm, right=2.3cm} 
	%从此处开始执行这个页眉设置,直到遇到\restoregeometry,恢复到导言区的页面设置
	\begin{spacing}{1.25} %设置行距为1.25倍行距
		~\\ %首行空一行
		%======第一行=======================%
		\zihao{-4}{中图分类号:}
		%\zihao{4}{\kaiGB\textbf{\underline{(四号楷体加粗)}}}
		% 上面注释的这句,下划线随着文字改变,而我们需要固定下划线长度
		% 然后让文字在下划线上居中,选择在下划线上生成一个固定长度的盒子
		\zihao{4}{\kaiGB\textbf{\underline{\makebox[8em][c]{(四号楷体加粗)}}}}
		% 在下划线上生成一个长度为8个四号字长度,文字居中的盒子
		\hspace{\stretch{1}}
		\zihao{-4}{密\qquad 级:}
		\zihao{4}{\kaiGB\textbf{\underline{\makebox[8em][c]{(四号楷体加粗)}}}}\\
		%========第二行=====================%	
		\zihao{-4}{UDC:}
		\zihao{4}{\kaiGB\textbf{\underline{\makebox[10em][c]{(四号楷体加粗)}}}}
		\hspace{\stretch{1}}
		\zihao{-4}{本校编号:}
		\zihao{4}{\kaiGB\textbf{\underline{\makebox[8em][c]{(四号楷体加粗)}}}}	
		\\ ~\\ 
		%按道理,~\\就是空一行,但是没有换行符它不空一行
		%若同时空多行例如:3行,则\\ ~\\ ~\\ ~\\,注意,每个~\\占一行
		\kaiGB 理由
		\zihao{3}{我的生活}
		\underline{\hbox to 40mm{我的理想}}\\
		\uline{1)	Supervisor \hfill }\\
		\underline{\makebox[7em][c]{\zihao{3}{我的理想是什么}}}
	\end{spacing}
	%\restoregeometry
	\section{ceng}
	
\end{document}