\documentclass[a4paper, 12pt, twoside]{article} %12pt是小四,奇偶页页眉不同,所以是双面
% 引入的宏包
\usepackage{ctex}
\usepackage{fancyhdr} %用来设置页眉页脚内容、样式等
\usepackage{geometry} %利用该宏包设置页面参数,包括页眉页脚的宽度等
\usepackage{setspace} %设置间距的宏包
\usepackage{ulem} %专门用来设置下划线相关的包,如uline,underline不在其中
\usepackage{graphicx} %用来插入图片的包
%\includeonly{FirstCover, SecondCover, ThirdCover}
\setCJKmainfont{宋体} %与\fontspec命令对应,只不过fontspec是指定英文的
\setmainfont{Times New Roman} %设置默认英文字体看,使得中文字体不更改英文ziti
\geometry{top=3.5cm, bottom=2.5cm, left=2.5cm, right=2.5cm, includeheadfoot, heightrounded}

%====================字体命令设置部分=============================%
%=====楷体GB2312设置=====%
\setCJKfamilyfont{kaiGB}[AutoFakeBold={3}]{楷体_GB2312} 
\newcommand{\kaiGB}{\CJKfamily{kaiGB}} %将命令重新定义,方便以后使用
%将楷体GB2312用kaiGB重命名,AutoFakeBold=3指调用\textbf加粗时字体多粗,默认值为3
%当不调用\textbf{}时,是不会加粗的
\setCJKfamilyfont{songti}[AutoFakeBold={3}]{SimSun}
\renewcommand{\songti}{\CJKfamily{songti}}

%====================end=========================================%
%============以下部分都是fancyhdr包中的内容=======================%
\pagestyle{fancy}
\fancyhead[CO]{\zihao{5}{兰州交通大学}}
\fancyhead[CE]{\zihao{5}{论文题目}}
%\cfoot{-\thepage-} %设计页码的风格样式
\begin{document}
	%==================第一页封面================%
		\newgeometry{top=2cm, bottom=2cm, left=3.7cm, right=2.3cm, heightrounded} 
	\thispagestyle{empty} %仅定义本页不要页眉页脚,不影响后面的页
	
	\vspace*{0\baselineskip} %设置行距对应word的1.25倍行距首行空行
	%======第一行=======================%
	\setstretch{1.345}\noindent\zihao{-4}{中图分类号:}
	%\zihao{4}{\kaiGB\textbf{\underline{(四号楷体加粗)}}}
	% 上面注释的这句,下划线随着文字改变,而我们需要固定下划线长度
	% 然后让文字在下划线上居中,选择在下划线上生成一个固定长度的盒子
	\zihao{4}{\kaiGB\textbf{\underline{\makebox[8em][c]{(四号楷体加粗)}}}}
	% 在下划线上生成一个长度为8个四号字长度,文字居中的盒子
	\hspace{\stretch{1}} %水平空格,直到占满整行
	\zihao{-4}{密\qquad 级:}
	\zihao{4}{\kaiGB\textbf{\underline{\makebox[8em][c]{(四号楷体加粗)}}}}
	
	%========第二行=====================%	
	\noindent\zihao{-4}{UDC:}
	\zihao{4}{\kaiGB\textbf{\underline{\makebox[10em][c]{(四号楷体加粗)}}}}
	\hspace{\stretch{1}}
	\zihao{-4}{本校编号:}
	\zihao{4}{\kaiGB\textbf{\underline{\makebox[8em][c]{(四号楷体加粗)}}}}
	
	%\setstretch{1.354} ~\\ ~\\
	\setstretch{2.48}\vspace{2\baselineskip}
	%==================第三行logo================%
	\begin{figure}[htbp]
		\centering
		\includegraphics*[scale=0.9]{Logo/logo.jpg}
	\end{figure}
	
	%\setstretch{1.354}~\\
	\setstretch{2.48}\vspace{1\baselineskip}
	%==================第四行学位论文类型========%
	\zihao{1}{\heiti\centerline{工\ 程\ 硕\ 士\ 学\ 位\ 论\ 文}}%黑体ctex包已经定义了,直接使用	
	%==============第五行:论文题目=============%
	\setstretch{1.354}~\\ ~\\
	
	%\setstretch{3.97}\vspace{2\baselineskip}
	\noindent\zihao{-3}{\kaiGB{论文题目:}}
	\zihao{2}{\kaiGB\underline{\makebox[16em][c]{(二号楷体)}}}\\
	\hspace*{3em} \zihao{2}{\kaiGB\underline{\makebox[16em][c]{题目比较长的第二行}}}\\
	\setstretch{1.354}~\\ ~\\
	%==============第六行:学生信息==================%
	\zihao{-3}{\kaiGB{研究生姓名:}}
	\zihao{-2}{\kaiGB\textbf{\underline{\makebox[8em][c]{(小二号楷体加黑)}}}}
	\hspace{\stretch{1}} 
	\zihao{-3}{\kaiGB{学号:}}
	\zihao{-2}{\kaiGB\textbf{\underline{\makebox[5em][c]{字体同姓名}}}}\\
	%===============第七行:学校导师信息============%
	\setstretch{1.354}~\\ ~\\
	\setstretch{2.167}\zihao{4}{\songti{学校指导教师姓名:}}
	\zihao{3}{\kaiGB\textbf{\underline{\makebox[7em][c]{三号楷体加黑}}}}
	\hspace{\stretch{1}}
	\zihao{4}{\songti{职称:}}
	\zihao{3}{\kaiGB\textbf{\underline{\makebox[6em][c]{副教授}}}}\\
	%=================第八行:企业导师信息===========%
	\zihao{4}{\songti{企业指导教师姓名:}}
	\zihao{3}{\kaiGB\textbf{\underline{\makebox[7em][c]{三号楷体加黑}}}}
	\hspace{\stretch{1}}
	\zihao{4}{\songti{职称:}}
	\zihao{3}{\kaiGB\textbf{\underline{\makebox[6em][c]{高工}}}}\\
	%================第九行:学位领域名称===========%
	\zihao{-3}{\songti{申请学位工程领域名称:}}
	\hspace{\stretch{1}}
	\zihao{3}{\kaiGB\textbf{\underline{\makebox[9.22cm][c]{三号楷体加黑}}}}
	%\zihao{3}{\makebox[][c]{\kaiGB\textbf{\underline{三号楷体加黑}}}}
	\hspace{\stretch{1}}\\
	%================第十行:论文提交日期及答辩日期=============%
	\zihao{4}{\songti{论文提交日期:}}
	\hspace{\stretch{1}}
	\zihao{-4}{\kaiGB\textbf{\underline{\makebox[9em][c]{三号楷体加黑}}}}
	\zihao{4}{\songti{论文答辩日期:}}
	\zihao{-4}{\kaiGB\textbf{\underline{\makebox[9em][c]{高工}}}}

	%============第二页封面=================%
	\newpage\thispagestyle{empty}

\setstretch{2.4375}\vspace*{0\baselineskip}
\setstretch{1.625}\zihao{3}{\centerline {独创性声明}}

\setstretch{1.354}\vspace{1\baselineskip}
本人声明所呈交的学位论文是本人在导师指导下进行的研究工作和取得的研究成果,除了文中特别加以标注和致谢之处外,论文中不包含其他人已经发表或撰写过的研究成果,也不包含获得\songti\textbf{\underline{\ 兰州交通大学\ }}或其他教育机构的学位或证书而使用过的材料。与我一同工作的同志对本研究所做的任何贡献均已在论文中作了明确的说明并表示了谢意。

\setstretch{1.354}\vspace{2\baselineskip}
\noindent 学位论文作者签名:\hspace{8em}签字日期:\qquad\quad 年\qquad 月\qquad 日

\setstretch{1.354}\vspace{6\baselineskip}
\setstretch{1.354}\zihao{3}{\centerline {学位论文版权使用授权书}}

\setstretch{1.8}\vspace{1\baselineskip}	
本学位论文作者完全了解\songti\textbf{\underline{\ 兰州交通大学\ }}有关保留、使用学位论文的规定。特授权\songti\textbf{\underline{\ 兰州交通大学\ }}可以将学位论文的全部或部分内容编入有关数据库进行检索,并采用影印、缩印或扫描等复制手段保存、汇编以供查阅和借阅。同意学校向国家有关部门或机构送交论文的复印件和磁盘。

\noindent(保密的学位论文在解密后适用本授权说明)

\setstretch{1.354}\vspace{3\baselineskip}
\noindent 学位论文作者签名:\hspace{11em}导师签名:

\noindent 签字日期:\qquad\quad 年\qquad 月\qquad 日\hspace{5em}签字日期:\qquad\quad 年\qquad 月\qquad 日
	%================第三页封面===============%
	\newgeometry{top=3.5cm, bottom=2.5cm, left=2.5cm, right=2.5cm}
%这一页与全局页面设计一样,单独再设一遍的原因是这一页没有页眉页脚。
\cleardoublepage
\thispagestyle{empty}

\setstretch{1.354}\vspace*{0\baselineskip}%实现1.25 字号为小四(默认)的空行
\centerline{\zihao{-2}{\textbf{工\ 程\ 硕\ 士\ 学\ 位\ 论\ 文}}}	

\setstretch{1.354}\vspace{2\baselineskip}
\setstretch{2.48}\vspace{1\baselineskip}
\setstretch{2.48}\centerline{\songti\textbf{\zihao{2}{兰州交通大学硕士学位论文格式规范}}}

\setstretch{1.8}\centerline{\textbf{\zihao{3}{The Format Criterion of Master Degree Thesis of LZJTU}}}

\setstretch{1.354}\vspace{13\baselineskip}
\begin{center}
	\zihao{4}作~者~姓~名:\underline{\makebox[11em][c]{名字}}\\
	工~程~领~域:\underline{\makebox[11em][c]{名字}}\\
	研~究~方~向:\underline{\makebox[11em][c]{名字}}\\
	学~~~~~~~~~~~号:\underline{\makebox[11em][c]{名字}}\\
	校~内~导~师:\underline{\makebox[11em][c]{名字}}\\
	企~业~导~师:\underline{\makebox[11em][c]{名字}}\\
	完~成~日~期:\underline{\makebox[11em][c]{名字}}
	
	\setstretch{1.354}\vspace{1\baselineskip}
	\setstretch{1.354}~\\
	\zihao{4}兰~~州~~交~~通~~大~~学\\
	Lanzhou Jiaotong University	
	\end{center}
\end{document}